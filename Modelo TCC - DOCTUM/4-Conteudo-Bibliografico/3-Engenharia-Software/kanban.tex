\subsubsection{\textit{\textit{Kanban}}}

De forma a complementar o assunto, o \textit{Kanban} é um método ágil de desenvolvimento de software que permite a interação de várias áreas e membros do projeto por meio de cartões que contem o progresso de cada atividade. Cada cartão contem uma instrução a ser seguida pela área ou pelo integrante no qual foi designado para a atividade. Sendo assim, seu principal foco e fornecer um trabalho progressivo, apresentando as evoluções e dificuldades de forma clara e transparente, favorecendo uma cultura de melhoria contínua \cite{KANBAN2014}.

Sendo assim, o \textit{Kanban} tem um grande potencial em trabalho conjunto para a finalização de um item em especifico, justamente para que não ocorra nenhum gargalo na entrega de um item essencial para o trabalho do integrante ou da área seguinte. Outra grande característica é evitar ou diminuir o índice de trabalhos repetitivos que, por um eventual descuido, possa acontecer de desenvolvedores realizarem a mesma codificação de uma mesma função ou \textit{API - Application Programming Interface} (Interface de Programação de Aplicações), por exemplo.

O \textit{Kanban} foi criado pelo vice presidente da \textit{Toyota Motor Company}, o sr. Taiichi Ohno, no qual teve como principal objetivo o aumento do valor agregado entregue nas atividades de cada colaborador de sua equipe. Com este pensamento,  Ohno concluiu que as pilhas de materiais estocados e as filas de de espera era um “dinheiro parado” que a empresa \textit{Toyota} estava desperdiçando. Portando,   Ohno uniu os princípios do método \textit{just in time} (Determina que tudo deve ser feito na hora exata)  juntamente com o \textit{Jidoka} (Determina que corrigir o problema em si não é o bastante, e sim corrigir a origem do problema) para elaborar um metodo mais aprimorado de organização dentro da empresa \textit{Toyota}, denominado \textit{Kanban} \cite{TOYOTA1977}.

Atualmente, a metodologia ágil \textit{Kanban} é utilizada em diversas empresas para realizar o controle de desempenho de diversas área de atuação.