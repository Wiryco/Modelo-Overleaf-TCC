\subsection{{Biblioteca \textit{OpenCV}}}

Devido aos avanços em estudos científicos tecnológicos na área de visão computacional, o surgimento de bibliotecas que utilizam desta tecnologia evoluiu consideravelmente, proporcionando aos usuários maior versatilidade na escolha da melhor ferramenta para uma possível solução de seus problemas. Dentre as principais ferramentas que implementam algoritmos de visão computacional, pode-se citar as mais utilizadas: \textit{Matlab}, \textit{OpenCV} e \textit{scikit-image}. Este trabalho tem o objetivo de abordar apenas a biblioteca \textit{OpenCV}, no qual será utilizada para o desenvolvimento de uma possível resolução ao problema supracitado. A biblioteca \textit{Open Source} (Código aberto) está disponível no seu site oficial \citeonline{OpenCV}.

Desenvolvida pela Intel no ano 2000, escrita nativamente em C++, a biblioteca OpenCV permite a manipulação de dados de imagens, manipulações vetoriais, rotinas de álgebra linear, desenvolvimento de algoritmos de processamento de imagem, calibração de câmeras, dentre outros. Sua flexibilidade com várias linguagens de programação, como por exemplo o \textit{Python}, permite uma melhor integração com vários programas, evitando possíveis conflitos de incompatibilidade e proporcionando uma melhor flexibilidade no desenvolvimento de \textit{softwares} \cite{BARBOZA2009}.

A bilioteca possui certificação BSD - \textit{Berkeley Software Distribution}, representando que o software possui uma licença gratuita. A biblioteca contem mais de 2.500 algoritmos otimizados com diversas propriedades para resolverem problemas extensos. Há também vários setores de aplicação da biblioteca, visto que esta abrange diversas áreas como, por exemplo, reconhecimento de face e objetos, extração de modelos de objetos tridimensionais, união de imagens em uma única imagem, pesquisar por imagens semelhantes dentro de um banco de dados, acompanhar movimentos dos olhos, reconhecimento de cenários, dentre outros. No site oficial da ferramenta encontra-se dados nos quais informam que esta possui uma comunidade com mais de 47 mil usuários e um número estimado de download que ultrapassa a casa dos 18 milhões \cite{CUNHA2013}.

Segundo \citeonline{CUNHA2013}, um dos objetivos da biblioteca \textit{OpenCV} é fornecer uma infraestrutura robusta na área de visão computacional na qual seja de fácil manipulação, ajudando os desenvolvedores no processo de ampliação de aplicações sofisticadas de visão.