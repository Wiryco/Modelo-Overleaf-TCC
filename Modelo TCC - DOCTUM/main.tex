\documentclass[
	% -- opções da classe memoir --
	12pt,				% tamanho da fonte
	openright,			% capítulos começam em pág ímpar (insere página vazia caso preciso)
	twoside,			% para impressão em recto e verso. Oposto a oneside
	a4paper,			% tamanho do papel. 
	% -- opções da classe abntex2 --
	chapter=TITLE,		% títulos de capítulos convertidos em letras maiúsculas
	%section=TITLE,		% títulos de seções convertidos em letras maiúsculas
	%subsection=TITLE,	% títulos de subseções convertidos em letras maiúsculas
	%subsubsection=TITLE,% títulos de subsubseções convertidos em letras maiúsculas
	% -- opções do pacote babel --
	english,			% idioma adicional para hifenização
	french,				% idioma adicional para hifenização
	spanish,			% idioma adicional para hifenização
	brazil				% o último idioma é o principal do documento
	]{abntex2}

%Importa todos os pacotes básicos para a estruturação do projeto

% ---
% Pacotes básicos 
% ---
\usepackage{lmodern}			% Usa a fonte Latin Modern			
\usepackage[T1]{fontenc}		% Selecao de codigos de fonte.
\usepackage[utf8]{inputenc}		% Codificacao do documento (conversão automática dos acentos)
%\usepackage{indentfirst}		% Indenta o primeiro parágrafo de cada seção.
\usepackage{color}				% Controle das cores
\usepackage{graphicx}			% Inclusão de gráficos
\usepackage{microtype} 			% para melhorias de justificação
\usepackage{multicol}
\usepackage{color}
\usepackage{layouts}
%\usepackage{float}

\usepackage[dvipsnames*,svgnames]{xcolor}
\usepackage[most]{tcolorbox}

\usepackage{doctum}
% ---

\usepackage[scaled]{helvet}
\renewcommand*\familydefault{\sfdefault}		
% ---
% Pacotes adicionais, usados apenas no âmbito do Modelo Canônico do abnteX2
% ---
\usepackage{lipsum}				% para geração de dummy text
% ---

% ---
% Pacotes de citações
% ---
\usepackage[brazilian,hyperpageref]{backref}	 % Paginas com as citações na bibl
\usepackage[abnt-emphasize=bf,alf]{abntex2cite}	% Título da citação em negrito

% --- 
% CONFIGURAÇÕES DE PACOTES
% --- 
\setlrmarginsandblock{3cm}{2cm}{*}
\setulmarginsandblock{3cm}{2cm}{*}
\checkandfixthelayout

% ---
% Configurações do pacote backref
% Usado sem a opção hyperpageref de backref
\renewcommand{\backrefpagesname}{Citado na(s) página(s):~}
% Texto padrão antes do número das páginas
\renewcommand{\backref}{}
% Define os textos da citação
\renewcommand*{\backrefalt}[4]{
	\ifcase #1 %
		Nenhuma citação no texto.%
	\or
		Citado na página #2.%
	\else
		Citado #1 vezes nas páginas #2.%
	\fi}%
% ---

%-------------------------------------------------

% ---
% Informações de dados para CAPA e FOLHA DE ROSTO
% ---
\titulo{Visão computacional aplicada no reconhecimento de jogadores de futebol americano}

\autor{THAYRONE MARQUES SILVA
 \\	VINÍCIUS ANDRADE LOPES}
 
\local{IPATINGA, MG}

\data{2019}

\orientador{Esp. Tales Wallace Souza}
% \coorientador{Equipe \abnTeX}
\instituicao{%
  FACULDADES DOCTUM DE IPATINGA
  \par
  BACHARELADO EM SISTEMAS DE INFORMAÇÃO}
\tipotrabalho{Trabalho de Conclusão de Curso}
% O preambulo deve conter o tipo do trabalho, o objetivo, 
% o nome da instituição e a área de concentração 
\preambulo{Este Trabalho de Conclusão de Curso foi julgado e aprovado, como requisito parcial a obtenção do título de bacharel em Sistemas de Informação na Faculdade Doctum de Ipatinga – Rede Doctum de Ensino, em 2018.}
% ---


% ---
% Configurações de aparência do PDF final

% alterando o aspecto da cor azul
\definecolor{blue}{RGB}{41,5,195}

% informações do PDF
\makeatletter
\hypersetup{
     	%pagebackref=true,
		pdftitle={\@title}, 
		pdfauthor={\@author},
    	pdfsubject={\imprimirpreambulo},
	    pdfcreator={LaTeX with abnTeX2},
		pdfkeywords={abnt}{latex}{abntex}{abntex2}{trabalho acadêmico}, 
		colorlinks=true,       		% false: boxed links; true: colored links
    	linkcolor=blue,          	% color of internal links
    	citecolor=blue,        		% color of links to bibliography
    	filecolor=magenta,      		% color of file links
		urlcolor=blue,
		bookmarksdepth=4
}
\makeatother
% --- 

% ---
% Posiciona figuras e tabelas no topo da página quando adicionadas sozinhas
% em um página em branco. Ver https://github.com/abntex/abntex2/issues/170
\makeatletter
\setlength{\@fptop}{5pt} % Set distance from top of page to first float
\makeatother
% ---

% ---
% Possibilita criação de Quadros e Lista de quadros.
% Ver https://github.com/abntex/abntex2/issues/176
%
\newcommand{\quadroname}{Quadro}
\newcommand{\listofquadrosname}{Lista de quadros}

\newfloat[chapter]{quadro}{loq}{\quadroname}
\newlistof{listofquadros}{loq}{\listofquadrosname}
\newlistentry{quadro}{loq}{0}

% configurações para atender às regras da ABNT
\setfloatadjustment{quadro}{\centering}
\counterwithout{quadro}{chapter}
\renewcommand{\cftquadroname}{\quadroname\space} 
\renewcommand*{\cftquadroaftersnum}{\hfill--\hfill}

\setfloatlocations{quadro}{hbtp} % Ver https://github.com/abntex/abntex2/issues/176
% ---

% --- 
% Espaçamentos entre linhas e parágrafos 
% --- 

% O tamanho do parágrafo é dado por:
\setlength{\parindent}{0cm}

% Controle do espaçamento entre um parágrafo e outro:
\setlength{\parskip}{1.5em}  % tente também \onelineskip

% ---
% Compila o indice
% ---
\makeindex
% ---


% ----
% Início do documento
% ----
\begin{document}
% Seleciona o idioma do documento (conforme pacotes do babel)
%\selectlanguage{english}
\selectlanguage{brazil}

% Retira espaço extra obsoleto entre as frases.
\frenchspacing 

% ----------------------------------------------------------
% ELEMENTOS PRÉ-TEXTUAIS
% ----------------------------------------------------------
% \pretextual

% ---
% Capa
% ---
\imprimircapa
% ---

% ---
% Folha de rosto
% (o * indica que haverá a ficha bibliográfica)
% ---
\imprimirfolhaderosto*
% ---

% ---
% Inserir a ficha bibliografica
% ---

% Isto é um exemplo de Ficha Catalográfica, ou ``Dados internacionais de
% catalogação-na-publicação''. Você pode utilizar este modelo como referência. 
% Porém, provavelmente a biblioteca da sua universidade lhe fornecerá um PDF
% com a ficha catalográfica definitiva após a defesa do trabalho. Quando estiver
% com o documento, salve-o como PDF no diretório do seu projeto e substitua todo
% o conteúdo de implementação deste arquivo pelo comando abaixo:

\input{1-Estrutura-do-Trabalho/ficha-catalografica.tex}
% ---

% ---
% Inserir folha de aprovação
% ---

% Isto é um exemplo de Folha de aprovação, elemento obrigatório da NBR
% 14724/2011 (seção 4.2.1.3). Você pode utilizar este modelo até a aprovação
% do trabalho. Após isso, substitua todo o conteúdo deste arquivo por uma
% imagem da página assinada pela banca com o comando abaixo:
%
% \begin{folhadeaprovacao}
% \includepdf{folhadeaprovacao_final.pdf}
% \end{folhadeaprovacao}
%
\setlength{\ABNTEXsignwidth}{12cm}
\begin{folhadeaprovacao}

  \begin{center}
    {\ABNTEXchapterfont\large\imprimirautor}

    \vspace*{\fill}\vspace*{\fill}
    \begin{center}
      \ABNTEXchapterfont\bfseries\Large\imprimirtitulo
    \end{center}
    \vspace*{\fill}
    
    \hspace{.45\textwidth}
    \begin{minipage}{.5\textwidth}
        \imprimirpreambulo
        
        \vspace{1cm}
        
        Média Final: \rule{3cm}{0.2mm}
    \end{minipage}%
    \vspace*{\fill}
   \end{center}
        
   Ipatinga, XX de dezembro de 20XX
   
   \vspace{1cm}
   
   \centering\textbf{Banca Examinadora}

   \assinatura{Prof. Orientador: \imprimirorientador \\ MBA em Gerenciamento de Projetos Doctum \\ Instituto Superior Doctum de Ipatinga
   } 
   \assinatura{Prof.ª Convidada: Maíza Cristina de Souza Dias \\ Mestre em Informática PUC-Minas \\ Instituto Superior Doctum de Ipatinga}
   \assinatura{Prof. Convidado: Giovani Scarabelli Silva \\ Especialista em Administração de Sistema de Informação - UFLA \\ Instituto Superior Doctum de Ipatinga}
   %\assinatura{\textbf{Professor} \\ Convidado 3}
   %\assinatura{\textbf{Professor} \\ Convidado 4} 
\end{folhadeaprovacao}
% ---

% ---
% Dedicatória
% ---
\begin{dedicatoria}[DEDICATÓRIA]
Dedico este trabalho primeiramente a Deus que nos sustentou durante toda a jornada. A nossos familiares e entes queridos, que acreditaram desde o início em nós.
\end{dedicatoria}
% ---

% ---
% Agradecimentos
% ---
\begin{agradecimentos}[AGRADECIMENTOS]
Agradecemos primeiramente a Deus que esteve conosco em todas as tribulações de nossas vidas, nos ajudando a superá-las.

Aos nossos pais, por todo o amor que nos deram, além da educação, ensinamentos e apoio. A nossos entes queridos que estiveram conosco durante toda esta caminhada.

A todo o corpo docente da Faculdade DOCTUM de Ipatinga que esteve presente em todos os momentos de nossos estudos, em especial para o nosso professor orientador Tales Wallace Souza, que nos auxiliou com todo seu profissionalismo para o desenvolvimento deste trabalho e a Maíza Cristina de Souza Dias, que contribuiu com seu conhecimento e especialidade sobre o assunto decorrente deste trabalho.

E finalmente, mas não menos importante, a todos os integrantes da turma de sistemas de informação, que percorreram toda esta jornada conosco.
\end{agradecimentos}
% ---

% ---
% Epígrafe
% ---
\begin{epigrafe}
\vspace*{\fill}
\begin{flushright}
\textit{Quando todos nos unirmos contra as \\
injustiças e em defesa da privacidade e \\
dos direitos humanos básicos, poderemos nos \\
defender até dos mais poderosos dos sistemas. \\}
\vspace{3mm}
\textit{Edward Snowden}
\end{flushright}
\end{epigrafe}
% ---

% ---
% RESUMOS
% ---

% resumo em português
\setlength{\absparsep}{18pt} % ajusta o espaçamento dos parágrafos do resumo
%resumo em português
\begin{resumo}[RESUMO]
A visão computacional vem sendo utilizada em larga escala, sendo difundida em várias áreas no mercado atual. Indústria e áreas desportivas são exemplos de áreas que têm apresentado um crescimento expressivo devido as soluções empregadas pela tecnologia citada. No entanto, a implantação deste tipo de tecnologia requer um conhecimento avançado sobre esta e também sobre as regras de negócios que serão aplicadas para a resolução de algum problema em específico. Como consequência disto, o investimento em \textot{hardwares} que são capazes de realizar altos níveis de processamento também deve ser levando em consideração, visto que um computador básico não tem potência suficiente para realizar análises de imagem em tempo real. Sendo assim, esta pesquisa exploratória tem a finalidade de apresentar, de maneira didática, o funcionamento da tecnologia de visão computacional aplicada no esporte futebol americano onde, a princípio, foi proposto o reconhecimento de um dos jogadores dentro de campo.

 \textbf{Palavras-chave}: Visão computacional. Processamento de imagem. Reconhecimento. Tecnologia.
\end{resumo}

% resumo em inglês
\begin{resumo}[ABSTRACT]
 \begin{otherlanguage*}{english}
Computer vision has been used on a large scale and is widespread in many areas of the current market. Industry and sports are examples of areas that have shown significant growth due to the solutions employed by the technology mentioned. However, deploying this type of technology requires advanced knowledge of this technology as well as the business rules that will be applied to solve any particular problem. As a consequence of this, the investment in hardware that is capable of performing high processing levels must also be taken into consideration, as a basic computer does not have enough power to perform real-time image analysis. Thus, this exploratory research has the purpose of presenting, in a didactic way, the operation of computer vision technology applied in the football sport where, at first, the recognition of one of the players on the field was proposed.

   \noindent 
   \textbf{Keywords}: Computer Vision. Image Processing. Recognition. Technology.
 \end{otherlanguage*}
\end{resumo}
% ---

% ---
% inserir lista de ilustrações
% ---
\pdfbookmark[0]{\listfigurename}{lof}
\listoffigures*
\cleardoublepage
% ---

% ---
% inserir lista de quadros
% ---
\pdfbookmark[0]{\listofquadrosname}{loq}
\listofquadros*
\cleardoublepage
% ---

% ---
% inserir lista de tabelas
% ---
\pdfbookmark[0]{\listtablename}{lot}
\listoftables*
\cleardoublepage
% ---

% ---
% inserir lista de abreviaturas e siglas
% ---
\begin{siglas}
  \item[2-D] Duas dimensões
  \item[API] \textit{Application Programming Interface}
  \item[APS] \textit{Active Pixel Sensor}
  \item[BSD] \textit{Berkeley Software Distribution}
  \item[CCD] \textit{Charge Coupled Devic}
  \item[CIE] Comissão Internacional de Iluminação
  \item[CMOS] \textit{Complementary Metal Oxide Semiconducto}
  \item[IDE] \textit{Integrated Development Environment}
  \item[MOS] \textit{Metal Oxide Semiconductor}
  \item[NASA] \textit{National Aeronautics and Space Administration}
  \item[NFL] \textit{National Football League}
  \item[OpenCV] \textit{Open Source Computer Vision Library}
  \item[RCS] \textit{Revision Control System}
  \item[RGB] \textit{Red (Vermelho), Green (Verde) e Blue (Azul)}
  \item[SCM] \textit{Source Code Manager}
  \item[VCS] \textit{Version Control System}
\end{siglas}
% ---

% ---
% inserir lista de símbolos
% ---
%\input{1-Estrutura-do-Trabalho/lista-simbolos.tex}
% ---

% ---
% inserir o sumario
% ---
\pdfbookmark[0]{\contentsname}{toc}
\tableofcontents*
\cleardoublepage
% ---



% ----------------------------------------------------------
% ELEMENTOS TEXTUAIS
% ----------------------------------------------------------
\textual

\pagestyle{simple}

\input{3-Introducao/introducao.tex}

% ----------------------------------------------------------
% PARTE
% ----------------------------------------------------------
%TODO TWS REMOVER PART \part{Preparação da pesquisa}
% ----------------------------------------------------------

%-- Inicio conteúdo bibliográficoo --%

\chapter{\textbf{FUNDAMENTOS CONCEITUAIS}}
\label{cap-fundamentos-conceituais}

Descrever a estrutura do referencial teórico com os topicos principais do projeto.

\input{4-Conteudo-Bibliografico/1-Futebol-Americano/futebol-americano.tex}

\section{\textbf{{Visão computacional}}}
\label{visao-computacional}

A visão computacional evoluiu consideravelmente nos últimos anos. Em consequência dessa evolução, a visão computacional se aprimorou a ponto de chegar mais próximo da visão humana, com a capacidade de maior eficiência em várias situações.

A visão computacional abrange todas as técnicas e métodos de processamento de imagem em um único meio, com o objetivo de ser mais eficiente nas análises de dados e informações compostas dentro de uma imagem. \citeonline{SILVA2017} contextualizam que algoritmos de visão computacional utiliza matrizes bidimensionais ou hiperdimensionais como entrada de dados e, a partir desta, produzem informações compactadas como saída.

De forma didática, a área de visão computacional utiliza modelos descritivos de objetos, pessoas e/ou cenas capturadas digitalmente para realizar tomadas de decisões, a fim de automatizar processos.

Segundo \citeonline{REHEM2009}, devido ao avanço tecnológico, desenvolveram-se computadores com maior capacidade de processamento gráfico, proporcionando ferramentas com um potencial maior na área de visão computacional. Pode-se dizer que essas ferramentas são bibliotecas onde o seu código fonte é constituído de um agrupamento de funções que potencializam o processamento gráfico de imagens e vídeos. Sendo assim, ao utilizar essas bibliotecas, tem-se a possibilidade de desenvolvimento de técnicas de aperfeiçoamento gráfico para realizar rastreamento de movimentos e de características humanas em tempo real.

Contudo, a visão computacional foi desenvolvida, segundo \citeonline{MARR76}, através da neurofisiologia da visão humana(\autoref{fig_visao-humana}). Seu modelo era estabelecido em níveis de compreensões necessárias à computação da visão estereoscópica, ou seja, o modelo é capaz de trabalhar com eficiência no ambiente tridimensional, onde a análise é feita através de duas imagens obtidas em postos diferentes. \citeonline{PERONTI2008} explica em seu artigo que estereoscopia é a visualização feita por dois mecanismos de captura de imagem em um mesmo foco (\autoref{fig_visao-computacional-camera}).

\begin{figure}[h]
	\caption{\label{fig_visao-humana}Esquema mostrando as imagens captadas em cada olho (par estereoscópico) e a imagem resultado da fusão deste par estereoscópico \cite{PERONTI2008}.}
	\begin{center}
		\includegraphics[scale=0.5]{4-Conteudo-Bibliografico/2-Visao-Computacional/Imagens-Visao-Computacional/visao-humana.png}
	\end{center}
	\centering \legend{Disponível em: https://i.imgur.com/Mg7u9eF.jpg}
\end{figure}

\begin{figure}[h]
	\caption{\label{fig_visao-computacional-camera}Mecanismos para captação de imagens com focos visuais coincidentes.}
	\begin{center}
		\includegraphics[scale=0.5]{4-Conteudo-Bibliografico/2-Visao-Computacional/Imagens-Visao-Computacional/visao-computacional-camera.png}
	\end{center}
	\centering \legend{Fonte: \cite{PERONTI2008}}
\end{figure}

\clearpage

\input{4-Conteudo-Bibliografico/2-Visao-Computacional/captura-identificacao-imagem.tex}

\subsection{Processamento de imagem}

De forma complementar ao assunto supracitado, o processamento de imagem evoluiu drasticamente no decorrer do avanço tecnológico. Portanto, há grande interesse nessa tecnologia pelo fato de proporcionar um grande número de aplicações, como por exemplo o aprimoramento de informações relativo a imagens para interpretação humana e análise automáticas por computador de informações extraídas da imagem capturada.

Contudo, o grande marco da área de processamento de imagens aconteceu no século XX. Com o surgimento dos primeiros computadores digitais com grande capacidade de processamento e o início do programa espacial norte-americano, ocorreu um grande impulso na área de processamento de imagem. O uso de técnicas computacionais de aprimoramento de imagens teve início no \textit{Jet Propulsion Laboratory} (Laboratório de Propulsão a Jato), localizado no centro tecnológico da NASA, em 1964, quando "imagens da lua transmitidas por uma sonda Ranger eram processadas por computador para corrigir vários tipos de distorção inerentes à câmera de TV acoplada à sonda". Essa tecnologia foi usada em grandes expedições tripuladas, como a Apollo \cite{FILHO1999}.

Para realizar o processamento de uma imagem, são definidos passos a serem seguidos para a garantia do objetivo final. A captura da imagem consiste no uso de dispositivos físicos sensíveis a espectros de energia eletromagnética que convertem o sinal elétrico para um formato digital. O pré-processamento consiste no realce da imagem para enfatizar características de interesse ou recuperar imagens que sofreram alguma degradação devido à introdução de ruído, perda de contraste ou borramento. A segmentação é a extração ou identificação dos objetos contidos na imagem, separando a imagem em regiões. Por fim, a classificação, é o processo que identifica a imagem observada \cite{GONZALEZ2002}.

As etapas básicas do processamento de imagem estão representadas através da \autoref{fig_etapas-processamento-imagem}.

\begin{figure}[h]
	\caption{\label{fig_etapas-processamento-imagem}Etapas básicas do processamento de imagens.}
	\begin{center}
		\includegraphics[scale=0.5]{4-Conteudo-Bibliografico/2-Visao-Computacional/Imagens-Visao-Computacional/etapas-processamento-imagem.jpg}
	\end{center}
	\centering \legend{Fonte: Adaptada de \citeonline{GONZALEZ2002}}
\end{figure}

Seres humanos conseguem distinguir vários padrões de cores com certa facilidade, levando em consideração que a análise é feita em um ambiente tridimensional. Na computação, esse discernimento de cores são mais complexos, independentemente da dimensão na qual a imagem será analisada. Isso ocorre porque vários fatores podem contribuir para a má performance computacional no tratamento da imagem, como por exemplo a falta ou excesso de luz, qualidade do sensor, qualidade da imagem, dentre outros.

\begin{quotation}
“Objetos que emitem luz visível são percebidos em função da soma das cores espectrais emitidas. Tal processo de formação é denominado aditivo. O processo aditivo pode ser interpretado como uma combinação variável em proporção de componentes monocromáticas nas faixas espectrais associadas às sensações de cor verde, vermelho e azul, as quais são responsáveis pela formação de todas as demais sensações de cores registradas pelo olho humano. Assim, as cores verde, vermelho e azul são ditas cores primárias. Este processo de geração suscitou a concepção de um modelo cromático denominado RGB (Red, Green, e Blue), para o qual a Comissão Internacional de Iluminação (CIE) estabeleceu as faixas de comprimento de onda das cores primárias.” \cite{QUEIROZ2006}
\end{quotation}

Cada \textit{pixel} é representado por um valor numérico que corresponde a sua cor em questão. Sendo assim, para que seja possível representar uma imagem em alguma escala de cor monocromática (preto e branco, ou escalas de cinza), basta associar o \textit{pixel} a um valor numérico relacionado a sua escala de tom.

A \autoref{fig_rep-pixel-monocromatico} exemplifica a associação dos \textit{pixels} de forma a obter uma imagem monocromática. Segundo \citeonline{NELMA2000}, para obter uma imagem em tons de cinzento, basta associar cada \textit{pixel} um valor inteiro não negativo de um byte, onde o valor 0 corresponde a cor preta e o valor máximo, 255, corresponde a cor branca. Os valores intermediários correspondem aos variados tons de cinza.

\begin{figure}[h]
	\caption{\label{fig_rep-pixel-monocromatico}Visão computacional de \textit{pixel} monocromático.}
	\begin{center}
		\includegraphics[scale=3.5]{4-Conteudo-Bibliografico/2-Visao-Computacional/Imagens-Visao-Computacional/rep-pixel-monocromatico.jpg}
	\end{center}
	\centering \legend{Disponível em: https://ai.stanford.edu/~syyeung/cvweb/tutorial1.html}
\end{figure}

Já as imagens coloridas exigem um poder maior de processamento para serem reconhecidas. Isso ocorre porque as imagens em RGB - \textit{Red, Green, and Blue} (Vermelho, Verde e Azul)  precisam de mais de uma banda para serem processadas, ou seja, são analisadas três matrizes de cores para formar a paleta de cor específica do tom capturado (\autoref{fig_rgb-representacao}). Depois da análise, são formadas as cores distintas que compõe a imagem.

\begin{figure}[h]
	\caption{\label{fig_rgb-representacao}Matriz de \textit{pixels} RGB.}
	\begin{center}
		\includegraphics[scale=0.3]{4-Conteudo-Bibliografico/2-Visao-Computacional/Imagens-Visao-Computacional/rgb-representacao.jpg}
	\end{center}
	\centering \legend{Fonte: Elaborada pelos autores do projeto.}
\end{figure}

De forma mais detalhada, \citeonline{LOPES2013} exemplificam que imagens coloridas também são imagens multibanda, ou multiespectral. As cores visíveis através de olhos humanos podem ser representadas pela combinação de bandas das cores primárias vermelha, verde e azul (\textit{Red, green} e \textit{blue}, respectivamente). A imagem colorida também pode ser armazenada por meio de imagens cromáticas e mapas de cores. Nesse caso, o valor de cinza de cada \textit{pixel} na imagem se torna um índice para uma entrada de mapa de cores, enquanto a entrada em si do mapa de cores contem os valores dos componentes referentes as tonalidades \textit{RGB} (\autoref{fig_rep-pixel-rgb}).

\begin{figure}[h]
	\caption{\label{fig_rep-pixel-rgb}Visão computacional de \textit{pixels} RGB.}
	\begin{center}
		\includegraphics[scale=0.4]{4-Conteudo-Bibliografico/2-Visao-Computacional/Imagens-Visao-Computacional/rep-pixel-rgb.png}
	\end{center}
	\centering \legend{Disponível em: https://www.analyticsindiamag.com/computer-vision-primer-how-ai-sees-an-image/}
\end{figure}

No entanto, a imagem capturada por algum dispositivo eletrônico pode chegar de forma irregular ate a parte de processamento. Essas falhas podem ser caracterizadas de várias formas, como por exemplo a presença de \textit{pixels} ruidosos, brilho e/ou contrastes desregulados, caracteres com dígitos incompletos ou apagados como em digitalizações de documentos.

A parte de processamento fica responsável por elaborar uma melhoria da imagem em questão, ajustando todos os parâmetros para que seja possível analisar precisamente todas as informações que estão disponíveis no arquivo de imagem. Sendo assim, por analogia as imagens processadas, trata-se de uma etapa que analisa de forma profunda todos os dados contidos na imagem, ou seja, a fase de processamento abrange os níveis mais baixos de análise de imagens, pois trabalham diretamente com valores de intensidade dos \textit{pixels}, visto que neste período não existe nenhuma informação relacionada a imagem para que seja possível facilitar o trabalho. O resultado desse processo gera imagens digitalizadas de qualidade melhor que a original.

\input{4-Conteudo-Bibliografico/2-Visao-Computacional/segmentacao-imagem.tex}

\subsubsection{\textit{Histograma}}

De tal forma, dando continuidade ao assunto, o histograma é um método que auxilia na identificação de objetos e/ou características específicas da imagem, obtendo uma maior precisão nos resultados obtidos.

Segundo \citeonline{FILHO1999}, histograma são conjuntos de vários números no qual são indicados os percentuais de \textit{pixels} de uma imagem que possui determinados níveis de cinza. Estes valores, normalmente representados por gráficos, apresentam, para cada nível de cinza, o seu percentual de \textit{pixels} correspondente na imagem. Com base nessa análise feita pelo histograma, pode-se obter os níveis de contraste, brilho, e ate mesmo informações de predominância clara ou escura.

\citeonline{FILHO1999} explicam que, através de equações matemáticas, é possível obter um resultado satisfatório ao analisar cada elemento deste conjunto. Este trabalho não tem por finalidade apresentar e/ou explicar cálculos matemáticos que cada função executa.

De forma a complementar o assunto, \citeonline{MAIZA2013} expressa em sua tese que, ao obter o histograma da imagem, pode-se alcançar medidas estatísticas dos níveis de cinza da imagem, como por exemplo o seu valor mínimo e máximo, valor médio, variância e desvio padrão. Portanto, o histograma seria como um método de probabilidade, onde o número de \textit{pixels} de um determinado nível de cinza pode ser utilizado para calcular um outro \textit{pixel} com o mesmo valor de cinza na imagem (\autoref{fig_histograma}).

\begin{figure}[h]
	\caption{\label{fig_histograma}Imagem (a) e seu respectivo histograma (b).}
	\begin{center}
		\includegraphics[scale=0.5]{4-Conteudo-Bibliografico/2-Visao-Computacional/Imagens-Visao-Computacional/histograma.png}
	\end{center}
	\centering \legend{Fonte: \cite{MAIZA2013}}
\end{figure}

\input{4-Conteudo-Bibliografico/2-Visao-Computacional/classificacao-de-imagem.tex}

\input{4-Conteudo-Bibliografico/2-Visao-Computacional/similaridade.tex}

\input{4-Conteudo-Bibliografico/2-Visao-Computacional/reconhecimento-facial.tex}

\section{\textbf{{Engenharia de \textit{software}}}}

Quando se pensa em desenvolvimento, manutenção, especificação e criação de um \textit{software}, pensa-se também em tecnologias e práticas de gerência de projetos para que a execução da aplicação aconteça de forma organizada, produtiva e com a máxima qualidade possível. Tudo isso esta contido em engenharia de \textit{software}.

Segundo \citeonline{PRESSMAN2016} em seu livro, um \textit{software} bem sucedido é aquele que atende a todos os requisitos do usuário, fica implementado durante um bom tempo, é de fácil manutenção e operabilidade. Por outro lado, um \textit{software} mal sucedido pode acarretar diversos fatos desagradáveis, levando os usuários a insatisfação e ao erro. 

Apesar de gerentes, lideres de projetos e profissionais envolvidos com a área técnica entenderem a necessidade de uma metodologia mais disciplinar no desenvolvimento de \textit{softwares}, existe ainda discursos de como e qual é a melhor metodologia a ser aplicada no projeto. Essa indecisão corre devido a grande demanda de produção que acontece atualmente, principalmente no setor de desenvolvimento de aplicações. Outra coisa que impacta negativamente é que profissionais e empresas começam a desenvolver \textit{softwares} de forma descontrolada mesmo com uma metodologia organizacional aplicada, justamente por não estarem preparados para uma abordagem disciplinar \cite{PRESSMAN2016}. 

Com base nisso, a engenharia de \textit{software} evoluiu rigorosamente, passando de uma simples técnica implementada por um publico relativamente pequeno para uma comunidade que objetiva o planejamento e a organização ates de iniciar qualquer tipo de desenvolvimento.

Sendo assim, em 2001, o engenheiro de \textit{software} Kent Beck juntamente com os principais
desenvolvedores de métodos ágeis, assinaram o “Manifesto para o Desenvolvimento Ágil de Software” \cite{SOMMERVILLE2011}, que tem por iniciativa a seguinte maneira:

\begin{quotation}
“Estamos descobrindo melhores maneiras de desenvolver \textit{softwares}, fazendo-o e ajudando outros a fazê-lo. Através desse trabalho, valorizamos mais:

\begin{itemize}
\item Indivfduos e interações do que processos e ferramentas.
\item \textit{Software} em funcionamento do que documentação abrangente.
\item Colaboração dos clientes acima de negociação contratual.
\item Respostas a mudanças acima de seguir um plano.
\item Ou seja, embora itens à direita sejam importantes, valorizamos mais os que estão à esquerda.\cite{SOMMERVILLE2011}"
\end{itemize}
\end{quotation}

\input{4-Conteudo-Bibliografico/3-Engenharia-Software/desenvolvimento-agil-de-software.tex}

\section{\textbf{Ferramentas de desenvolvimento do projeto}}

Em relação as ferramentas de desenvolvimento, é notório que existe um trajeto amplo a ser percorrido pelos programadores. Isso ocorre porque existem várias linguagens de programação e \textit{frameworks} que auxiliam no desenvolvimento de sistemas. A escolha da melhor linguagem de programação e o melhor \textit{framework} para desenvolvimento depende muito do problema a ser resolvido e também das habilidades do programador. Há também um grande impasse na escolha da melhor linguagem e/ou do melhor \textit{framework}, que está relacionado a: interesses e aplicações comerciais; comunidade, para sanar possíveis duvidas e curva de aprendizado.

\input{4-Conteudo-Bibliografico/4-Ferramentas-de-Desenvolvimento-do-Projeto/linguagem-programacao.tex}

\input{4-Conteudo-Bibliografico/4-Ferramentas-de-Desenvolvimento-do-Projeto/python.tex}

\input{4-Conteudo-Bibliografico/4-Ferramentas-de-Desenvolvimento-do-Projeto/framework.tex}

\subsection{{Ambientes virtuais}}

A virtualização vem sendo muito utilizada na área de desenvolvimento de \textit{software} devido a sua flexibilidade e fácil manipulação para realizar a criação de ambientes virtuais. Isso ocorre porque, com a virtualização, o usuário pode criar vários ambientes virtuais e, dentro deles, realizar a instalação das dependências necessárias para a execução do projeto. Portanto, o desenvolvedor pode ativar e desativar o ambiente virtual assim que necessário, sendo que as dependências do projeto estarão instaladas somente nesse ambiente, e não globalmente na máquina. Ou seja, as dependências do projeto serão desativadas assim que a virtualização for encerrada. A utilização de virtualização permite ainda que recursos computacionais possam ser alocados para múltiplas aplicações simultaneamente, sendo que cada uma dessas aplicações possui seu ambiente isolado das demais.

A situação citada acima é bastante válida se analisarmos o crescimento de ferramentas de programação disponíveis atualmente. Instalar vários \textit{frameworks}, pacotes de dependências de linguagens, bibliotecas e afins globalmente na máquina pode acarretar a lentidão, conflitos entre versões de linguagens de programação e \textit{frameworks} dos projetos, dentre outros. Quando isso acontece, por exemplo, em um projeto feito por uma equipe, a atualização de todas as dependências deve ser feita em todas as máquinas que estão envolvidas no projeto, para que a versão seja padrão em toda a equipe.

\input{4-Conteudo-Bibliografico/4-Ferramentas-de-Desenvolvimento-do-Projeto/versionamento.tex}

\input{4-Conteudo-Bibliografico/4-Ferramentas-de-Desenvolvimento-do-Projeto/open-cv.tex}

%-- Fim conteúdo bibliográfico --%

%-- Metodologia --%

\chapter{\textbf{METODOLOGIA}}

A metodologia utilizada no projeto é a exploratória.

Segundo \citeonline{VENTURA2007}, são verificadas grandes utilidades em estudos de casos realizados através de pesquisas exploratórias.

\begin{quotation}
“Por sua flexibilidade, é recomendável nas fases iniciais de uma investigação sobre temas complexos, para a construção de hipóteses ou reformulação do problema. Também se aplica com pertinência nas situações em que o objeto de estudo já é suficientemente conhecido a ponto de ser enquadrado em determinado tipo ideal. São úteis também na exploração de novos processos ou comportamentos, novas descobertas, porque têm a importante função de gerar hipóteses e construir teorias. Ou ainda, pelo fato de explorar casos atípicos ou extremos para melhor compreender os processos típicos \cite{VENTURA2007}.”
\end{quotation}

Sendo assim, será feito um estudo bibliográfico a partir do tema de visão computacional utilizando o acervo disponível sobre o assunto, com a finalidade de reunir informações para obter melhores resultados.

A pesquisa qualitativa será baseada principalmente nos três livros escolhidos: \textit{Digital Image Processing} \cite{GONZALEZ2002},  \textit{Computer Vision: Algorithms and Applications} \cite{SZELISKI2010} e Processamento Digital de Imagens \cite{FILHO1999}. Todos os livros foram escolhidos por transmitirem conceitos científicos, didáticos, teóricos e práticos sobre o assunto de visão computacional.

O desenvolvimento da aplicação proposta será feito através de estudos realizados sobre o tema. A aplicação será desenvolvida utilizando metodologia de desenvolvimento ágil, ou seja, será feita o planejamento de cada etapa de desenvolvimento do sistema, na qual o objetivo de cada entrega é proporcionar uma prévia do funcionamento do \textit{software}. Essa ação foi tomada para que a evolução do sistema seja gradativa e eficiente.

Para resumir a elaboração deste projeto, foi feito uma imagem ilustrativa da metodologia utilizada no decorrer das etapas de desenvolvimento deste (\autoref{fig_metodologia-desenvolvimento-tcc}).

\begin{figure}[h]
	\caption{\label{fig_metodologia-desenvolvimento-tcc}Metodologia de desenvolvimento do projeto.}
	\begin{center}
		\includegraphics[scale=0.5]{5-Metodologia/metodologia-desenvolvimento-tcc.png}
	\end{center}
	\centering \legend{Fonte: Elaborada pelos autores do projeto.}
\end{figure}

%-- Fim Metodologia --%

%-- Inicio conteúdo desenvolvimento --%

\chapter{\textbf{DESENVOLVIMENTO}}

Neste capítulo aborda todo o processo utilizado para o desenvolvimento desse projeto...

\section{\textbf{Descrição do sistema}}

O QUE O SOFTWARE VAI FAZER

\input{6-Desenvolvimento-Projeto/etapas-de-funcionamento.tex}

\section{\textbf{Desenvolvimento do sistema}}

COMO FOI FEITO

\input{6-Desenvolvimento-Projeto/requisitos-funcionais.tex}

\subsection{{Requisitos não-funcionais}}

\section{\textbf{Ambiente de testes}}

%-- Fim conteúdo desenvolvimento --%

\chapter{\textbf{CONSIDERAÇÕES FINAIS}}

Escreva aqui as conclusões deste trabalho, lembrando que os objetivos geral e específicos devem ser citados e comentados,  ou seja, descreva como cada objetivo foi alcançado.

\section{\textbf{Análise dos resultados}}

\section{\textbf{Propostas de novos estudos}}

% ----------------------------------------------------------

% ----------------------------------------------------------
% Finaliza a parte no bookmark do PDF
% para que se inicie o bookmark na raiz
% e adiciona espaço de parte no Sumário
% ----------------------------------------------------------
%\phantompart
% ----------------------------------------------------------
% ELEMENTOS PÓS-TEXTUAIS
% ----------------------------------------------------------
\postextual
% ----------------------------------------------------------

% ----------------------------------------------------------
% Referências bibliográficas
% ----------------------------------------------------------
%\bibliography{abntex2-modelo-references}
\bibliography{referencias}

% ----------------------------------------------------------
% Glossário
% ----------------------------------------------------------
%
% Consulte o manual da classe abntex2 para orientações sobre o glossário.
%
%\glossary

% ----------------------------------------------------------
% Apêndices
% ----------------------------------------------------------

% ---
% Inicia os apêndices
% ---
\begin{apendicesenv}
% Imprime uma página indicando o início dos apêndices
\partapendices

% ----------------------------------------------------------
\input{Apendice/apendice-teste.tex}
% ----------------------------------------------------------

\end{apendicesenv}
% ---


% ----------------------------------------------------------
% Anexos
% ----------------------------------------------------------

% ---
% Inicia os anexos
% ---
\begin{anexosenv}

% Imprime uma página indicando o início dos anexos
\partanexos

%-------------------------------------------------------------
\input{Anexos/anexo-teste.tex}
%-------------------------------------------------------------

\end{anexosenv}

%-------------------------------------------------------------
% INDICE REMISSIVO
%---------------------------------------------------------------------
\phantompart
\printindex
%---------------------------------------------------------------------

\end{document}
